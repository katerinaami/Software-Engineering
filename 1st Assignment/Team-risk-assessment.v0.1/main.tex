\documentclass{article}
\usepackage[utf8]{inputenc}
\usepackage{multirow}
\usepackage[greek,english]{babel}
\usepackage{alphabeta}
\usepackage{ragged2e}
\usepackage[topmargin=0.5in]{geometry}

\usepackage{graphicx}

\begin{document}



\begin{titlepage}
   \begin{center}
       \vspace*{1cm}

           \textbf{\LARGE{Team Risk Assessment-v0.1}}

       \vspace{0.5cm}
        
            
       \vspace{2cm}
        
       \large{Βεργίνης Δημήτριος}\\
       \large{Βλαχογιάννης Δημήτριος}\\
       \large{Κούρου Αγγελική} \\
       \large{Μητροπούλου Αικατερίνη} \\
       \large{Στεφανίδης Μάριος}\\
      \vspace{1.5cm}
   \end{center}
   \textbf{Περιγραφή}\\
   Στο παρακάτω κείμενο παρουσιάζουμε τους πιθανούς κινδύνους που μπορούν να προκύψουν κατά την διάρκεια εκπόνησης του έργου στο μάθημα Τεχνολογίας Λογισμικού στο πλαίσιο συνεργασίας της ομάδας.
   
\end{titlepage}

\begin{table}[]
\begin{tabular}{|ccccc|}
\hline
Κίνδυνος &
  Πιθανότητα εμφάνισης &
  Σοβαρότητα &
   &
  Μέθοδος Διαχείρησης Κινδύνου \\ \hline
\begin{tabular}[c]{@{}c@{}}Πιθανή αποχώρηση\\ κάποιου μέλους της\\ ομάδας κατά την διάρκεια\\ εκπόνησης του έργου.\end{tabular} &
  Χαμηλή &
  Μεσαία &
   &
  \begin{tabular}[c]{@{}c@{}}Συνέχιση εκπόνησης του\\ έργου με τα υπόλοιπα\\ άτομα όσο ο αριθμός\\ αυτός βρίσκεται από τρία\\ και πάνω.\end{tabular} \\ \hline
\begin{tabular}[c]{@{}c@{}}́Εντονη αντιπαράθεση\\ μεταξύ των μελών της\\ ομάδας.\end{tabular} &
  Χαμηλή &
  Μεσαία &
   &
  \begin{tabular}[c]{@{}c@{}}Διάλογος κατά την\\ διάρκεια των συναντήσεων\\ με στόχο την επίλυση της\\ τρέχουσας διένεξης.\end{tabular} \\ \hline
\begin{tabular}[c]{@{}c@{}}Προσωρινή αδυναμία \\ κάποιου ατόμου να \\ συνεισφέρει στο έργο \\ λόγω προσωπικού \\ προβλήματος (π.χ. νόσηση).\end{tabular} &
  Χαμηλή &
  Χαμηλή &
   &
  \begin{tabular}[c]{@{}c@{}}Διαφορετικός\\ διαμοιρασμός των\\ τρεχόντων εργασιών για\\ την διευκόλυνση του\\ συγκεκριμένου ατόμου.\end{tabular} \\ \hline
\begin{tabular}[c]{@{}c@{}}Έλλειψη γνώσεων ή\\ δεξιοτήτων από τα μέλη\\ της ομάδας και αδυναμία\\ ολοκλήρωσης των\\ εργασιών.\end{tabular} &
  Χαμηλή &
  Χαμηλή &
   &
  \begin{tabular}[c]{@{}c@{}}'Αμεση ομαδική\\ προσπάθεια κάλυψης της\\ συγκεκριμένης εργασίας\\ καθώς και προτροπή για\\ γρήγορη εφόσον είναι\\ εφικτή η κάλυψη του κενού.\end{tabular} \\ \hline
\begin{tabular}[c]{@{}c@{}}Έλειψη επικοινωνίας μεταξύ\\ των μελών της ομάδας.\end{tabular} &
  Χαμηλή &
  Μεσαία &
  \multicolumn{1}{l}{} &
  \begin{tabular}[c]{@{}c@{}}Παρέμβαση του project manager\\ είτε με περισσότερα meetings \\ είτε κάνοντας γενικότερη \\ ανασκόπηση της σημαντικότητας\\ ολοκλήρωσης του έργου.\end{tabular} \\ \hline
\begin{tabular}[c]{@{}c@{}}Άρνηση κάποιου μέλους\\ να αναλάβει την εργασία \\ που του αναλογεί.\end{tabular} &
  Χαμηλή &
  Χαμηλή &
  \multicolumn{1}{l}{} &
  \begin{tabular}[c]{@{}c@{}}Είτε καταγραφή μειωμένης \\ συνεισφοράς στο έργο είτε\\ διάλογος για την εύρεση της\\ αιτίας του προβλήματος.\end{tabular} \\ \hline
\end{tabular}
\end{table}

\end{document}
